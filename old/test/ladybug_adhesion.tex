%% LyX 2.3.5.2 created this file.  For more info, see http://www.lyx.org/.
%% Do not edit unless you really know what you are doing.
\documentclass[english]{achemso}
%\usepackage{biblatex}
%\usepackage[backend=biber]{biblatex}
%\usepackage[backend=biber,bibstyle=ACM-Reference-Format,citestyle=chem-rsc,url = false, isbn = false]{biblatex}
%\addbibresource{\string"E:/Work/Documents/PhD Literature.bib\string"}

\usepackage[T1]{fontenc}
\usepackage[utf8]{inputenc}
\usepackage{babel}
\usepackage{graphicx}
\usepackage[unicode=true,pdfusetitle,
 bookmarks=true,bookmarksnumbered=false,bookmarksopen=false,
 breaklinks=false,pdfborder={0 0 0},pdfborderstyle={},backref=false,colorlinks=false]
 {hyperref}
\setkeys{acs}{doi = true}

\makeatletter

%%%%%%%%%%%%%%%%%%%%%%%%%%%%%% LyX specific LaTeX commands.


\title{Shape of adhesive fluid controls insect adhesion in air and underwater}


\author{Pranav Sudersan}


\affiliation{Max Planck Institute for Polymer Research, Ackermannweg 10, 55128
Mainz, Germany}


\author{Thomas Endlein}


\affiliation{Max Planck Institute for Polymer Research, Ackermannweg 10, 55128
Mainz, Germany}


\author{Michael Kappl}


\affiliation{Max Planck Institute for Polymer Research, Ackermannweg 10, 55128
Mainz, Germany}


\author{Hans-Jürgen Butt}


\affiliation{Max Planck Institute for Polymer Research, Ackermannweg 10, 55128
Mainz, Germany}


\email{butt@mpip-mainz.mpg.de}


\phone{+49 6131 379 111}


\fax{+49 6131 379 310}


%%%%%%%%%%%%%%%%%%%%%%%%%%%%%% User specified LaTeX commands.
\SectionNumbersOn

\makeatother

%\usepackage[bibstyle=ACM-Reference-Format,citestyle=chem-rsc,url = false, isbn = false]{biblatex}
%\addbibresource{\string"E:/Work/Documents/PhD Literature.bib\string"}
\begin{document}
\begin{abstract}
Insects like beetles and flies can stick to various surfaces using
hairy pads mediated by adhesive fluid. The pads can even attach underwater,
presumably due to an air bubble trapped around the pad. There is however
a lack of understanding on the exact role of the bubble for underwater
adhesion. Here, we develop a simple theoretical model to estimate
the net adhesion of a hairy pad due to capillary forces. We perform\emph{
in-vivo} adhesion measurements of a constrained ladybug pad as well
as image its contact against smooth hydrophilic and hydrophobic substrates
in air and underwater conditions. Our experiments reveal that on hydrophobic
substrates, even without a bubble, the pad can show adhesion underwater
comparable to that in air. Only on hydrophilic substrates, a trapped
bubble is essential to generate adhesion underwater. Based on the
model, this observation is explained qualitatively. The shape of the
adhesive fluid has a negative curvature when on contact with a hydrophobic
substrate underwater, resulting in capillary forces due to the negative
Laplace pressure. Our results demonstrate that capillary forces governed
by the shape of the adhesive fluid well explains insect adhesion under
different conditions. 
\end{abstract}

\section{Introduction}

Many\textsuperscript{\cite{RN87,RN93}}

\section{Theoretical}

\subsection{Capillary Bridge Model}

This is a cool model

\begin{figure}

\includegraphics[scale=0.5]{\string"model_schematic\string".png}\caption{Model}

\end{figure}


\subsection{Results}

Wow! Nice results!

\begin{figure}

\includegraphics[scale=0.5]{\string"model_result-hydrophilic\string".png}

\includegraphics[scale=0.5]{\string"model_result-hydrophobic\string".png}\caption{Effect of substrate}

\end{figure}


\section{Experimental}

\subsection{Material and Methods}

I bought XYZ from PQR.

\subsubsection{Substrate preparation}

I made glass and fluorinate surface

\subsubsection{Contact angle measurement}

Settings for dynamic contact angle measurement

\subsubsection{Adhesion measurement}

Describe setup and protocol. I measure forces.

\subsubsection{Data analysis}

Describe image processing, statistical techniques.

\begin{figure}
\includegraphics[scale=0.5]{\string"setup_schematic\string".pdf}\caption{Setup}
\end{figure}


\subsection{Results}

Super results!

\begin{figure}
\includegraphics[scale=0.5]{\string"effect_of_contact-glass\string".jpg}

\includegraphics[scale=0.5]{\string"effect_of_contact-pfots\string".jpg}\caption{Effect of contact}
\end{figure}


\section{Discussion}

talk talk

\begin{figure}
\caption{Comparison with model}

\end{figure}

\begin{figure}

\caption{Oil contact images}

\end{figure}


\section{Conclusion}

that's all folks!

\mciteErrorOnUnknownfalse

\bibliography{references} %no file format, is in same dir.

%\printbibliography

\end{document}
